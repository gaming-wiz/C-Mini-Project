% ---- Don't modify from Line no. 2 to 74 ----
\documentclass[12pt]{article}

\usepackage{lineno,hyperref}
\modulolinenumbers[5]
\usepackage{graphics}
\usepackage{graphicx}
\usepackage{cite}
\usepackage{epsfig}
\usepackage{amsmath}   
\usepackage{amssymb}
\usepackage{placeins}
\usepackage[linesnumbered,ruled,vlined]{algorithm2e}
\usepackage{setspace}
\usepackage{multirow}
\usepackage[export]{adjustbox}[2011/08/13]
\usepackage{tabularx}
\usepackage{algcompatible}
\usepackage{caption}
\usepackage{epsf}
\usepackage{epstopdf}
\usepackage{subfigure} 
\usepackage{colortbl}
\usepackage{longtable}
\usepackage{enumerate}
\usepackage{tabularx, booktabs}

\usepackage[table,xcdraw]{xcolor}

\usepackage{tikz}
\usepackage{multirow}
\usepackage{enumitem}
\usepackage{soul}
\usepackage{xcolor}
\usepackage[utf8]{inputenc}
\usepackage{placeins}
\usepackage{makecell}
\newcounter{qcounter}
\usepackage{tcolorbox}
\usepackage{lscape}
\usepackage{url}
\usepackage{hyperref}
\usepackage{tablefootnote}
\usepackage{url}
\usepackage{geometry}
 \geometry{
 a4paper,
 total={170mm,257mm},
 left=20mm,
 top=20mm,
 }

\usepackage{hyperref}
\hypersetup{
    colorlinks=true,
    linkcolor=blue,
    filecolor=magenta,      
    urlcolor=cyan,
}

\setlength{\parindent}{4em}
\setlength{\parskip}{1em}
\renewcommand{\baselinestretch}{1.5}

\usepackage[numbers]{natbib}
\bibliographystyle{unsrtnat}

\begin{document}

% ------------ Don't modify anything up to here ---------------

% From here on-wards modify only the relevant fields, such as Title (line no. 76), "Section:", "Course Instructor:", and "Team Members:" field. (Team Members details should be in the format such as, name, reg. no., mobile no. and email id.). Further, "Title:" can be changed as per your selected topic name. In the brief description field you can describe your topic in 250 words. Additionally, in the "Key Feature:" field add your mini-project feature as an item (example is shown). AT the end in the "Reference:" field add the website/paper/article referred for this mini-project as an item.

\begin{center}
    \textbf{\Large{Abstract \\
    (\textcolor{red}{Car Rental System})}}
\end{center}

\noindent 
\textbf{Course Code:} CS110 
\hspace{2in} 
\textbf{Course Title:} Computer Programming \\
\textbf{Semester:} B. Tech 1$^{st}$ Sem 
\hspace{1.6in} 
\textbf{Section:} S3 \\
\textbf{Academic Year:} 2019-20 
\hspace{1.8in} 
\textbf{Course Instructor:} Vaishnavi T \\
\textbf{Team Members:} \\
\textbf{1.} Pratham N, 191MT034, 6363583848, prathamn452001@gmail.com 
\newline
\textbf{2.} Saransh Bhaduka, 191ME252, 8696966881, saranshbhaduka111@gmail.com
\newline
\textbf{3.} Sumanth N Hegde, 191ME284, 7019468097, mnphah@gmail.com
\newline
\textbf{4.} Prasanna Sarkar, 191MT033, 9472839780, sarkarprasanna2u@gmail.com

\vspace{0.25in}

\noindent
\textbf{Brief Description:}
\newline
Write your 250 words brief description about your mini-project topic here.

\noindent
\textbf{Key Features:}
\begin{enumerate}
    \item Login facility (New or existing account)
    \item Listing car models available
    \item Renting cars
    \item Printing invoice based on duration
    \item User Profile Display
    \item Returning the cars
    \item Feedback/rating facility
\end{enumerate}

\noindent
\textbf{References:}
\begin{enumerate}
    \item http://services.lovelycoding.org/hotel-management-system/
    \item https://1000projects.org/hotel-management-system-project.html
\end{enumerate}


\begin{center}
    \textbf{**** END ****}
\end{center}

\end{document}
